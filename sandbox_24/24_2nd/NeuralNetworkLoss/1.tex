% ドキュメントクラスの定義
\documentclass[twocolumn, a4j, 10pt, fleqn]{ltjsarticle}

\usepackage{enumitem}
\usepackage{url}
\usepackage{amsmath}
\usepackage[margin=10truemm]{geometry}
\usepackage{graphicx}
\usepackage{caption}

% tcolorbox関連
\usepackage{tcolorbox}
\tcbuselibrary{breakable, skins, theorems}

\captionsetup[figure]{format=plain, labelformat=simple, labelsep=period}

\setlength{\columnseprule}{0.1pt}
% \setlength{\mathindent}{0pt}
\renewcommand{\baselinestretch}{0.8}
\renewcommand{\figurename}{Fig}
\setlength{\textfloatsep}{0pt}

% 部名の変更
\renewcommand{\prepartname}{第}
\renewcommand{\postpartname}{部}
\renewcommand{\thepart}{\arabic{part}}

\begin{document}

\title{Neural Network Loss}
\author{93 san}
\maketitle

\part{ModelX2}
本部では、各節で説明される問題を解決したNeuralNetworkを提案する.

\section{DQN-MC loss改善}
DQN-MCでは、学習時にモデルが収束しない原因として、以下の二つの原因があげられる。NeuralNetwork自身は、与えられたサンプルに対して十分に学習が進んでおり、これ以上lossが下がる能力がない状況を前提とする.

\begin{tcolorbox}[
  colback = white,
  colframe = black,
  fonttitle = \bfseries,
  breakable = true]
\begin{itemize}
  \item $Q(s_t, a_1)$と$Q(s_t, a_2)$の真の期待値が十分近い場合、各NN更新ステップのミニバッチ学習時に、ミニバッチ内のターゲットの分布の偏りによって、NNが学習している$Q(s_t, a_1)$と$Q(s_t, a_2)$の大小関係が逆転することで、生成されるエピソードが変動する.
  \item NNの精度が十分でない場合、本来十分離れているはずの$Q(s_t, a_1)$と$Q(s_t, a_2)$の真の期待値を十分に予測できず、各NN更新ステップのミニバッチ学習時に、$Q(s_t, a_1)$に対する予測の振動幅と、$Q(s_t, a_2)$に対する予測の振動幅が重なることで、$Q(s_t, a_1)$と$Q(s_t, a_2)$の大小関係が逆転し、生成されるエピソードが変動する.
\end{itemize}
\end{tcolorbox}
\noindent
*ただし、$ReplayBuffer$内の各$(s, a)$の組に対するターゲットの期待値が変動しない仮定において.\\
\noindent
*ここで「収束しない」とは、学習中のモデルがgreedyに行動決定した場合でも、生成されるエピソードが変動することを指す.

\vskip\baselineskip
上記の原因二つに対して、改善可能と考えられるのは以下である.

\begin{enumerate}
  \item 報酬関数の変更:\\
        $Q(s_t, a_1)$と$Q(s_t, a_2)$の期待値の差を広げることで、大小関係の逆転を防ぐ. もしくは、完全に各$(s_t, a_k)$に対して、一意の報酬が与えられるような報酬体系にする(DQN-MCでは不可).
  \item NNのloss改善:\label{enum:NNloss}\\
        $Q(s_t, a_1)$と$Q(s_t, a_2)$の期待値が十分に離れていることを前提に、両者の大小関係が逆転しないレベルまでNNのlossを低下させる.
  \item Backup方法の変更:\\
        DQN-MC以外のBackup方法に変更することで、期待値を予測するという学習をやめる.
\end{enumerate}
本節では、上記の\ref{enum:NNloss}について取り組む.

\subsection{Neural Network Architecture For Decrease loss of ModelX2}

\part{ModelX3}

\end{document}