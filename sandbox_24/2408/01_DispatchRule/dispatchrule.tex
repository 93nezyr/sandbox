\documentclass[twocolumn]{ltjsarticle}
\usepackage{enumitem}
\usepackage{amsmath}

% \setlength{\mathindent}{0cm}

\begin{document}

\title{Dispatch Rule for JSP}
\author{93nezyr}
\maketitle

\section{shuluo2020}

\section{調査・提案手法}

調査中に登場した、ディスパッチルールに関する一般的な概念の定義.

\begin{description}[style=multiline, leftmargin=10em]
  \item[$SlackTime$] $SlackTime = DueDate - CurrentTime - WorkRemained$で定義される値. \\
  これって改行できる?
  \item[$P_j$] ジョブ$j$の処理時間
\end{description}

一般的に使われるディスパッチルールのうち、未だ実験に使用されていないのは以下.

\begin{enumerate}
  \item Longest Processing Time \\
        最も処理時間の長いジョブを優先してスケジュールする.
  \item Most Work Remaining \\
        残り作業量が最も多いジョブを優先してスケジュールする.
  \item Least Work Remaining \\
        残り作業量が最も少ないジョブを優先してスケジュールする.
\end{enumerate}

(1)もしかしてこうやって何か情報を書いて \\
式(N)に示すとか言うて:
\begin{align}
  y = f(x)
\end{align}

\begin{flalign}
  &y = f(x) &
\end{flalign}

\end{document}