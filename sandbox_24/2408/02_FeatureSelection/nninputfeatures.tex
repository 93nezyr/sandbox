\documentclass[twocolumn, a4j]{ltjsarticle}
\usepackage{enumitem}
\usepackage{url}

\setlength{\columnseprule}{0.1pt}

\begin{document}

\title{Neural Network Input Features}
\author{93nezyr}
\maketitle

\section{入力特徴量生成・抽出}

ジョブショップスケジューリング問題(JSSP)における、特徴量選択方法について議論する。今回のJSSPは、ガントチャートで表現されるものとする。ガントチャートの行方向をJSSPにおけるリソース(マシン)、列方向をスケジューリングの時刻とすると、JSSPの状態には以下の変数が含まれる。

\begin{description}[style=multiline, leftmargin=10em]
  \item[$t$] スケジューリング中の現在時刻
  \item[$R$] リソースの集合
  \item[$R_i \in R$] リソース$i$
  \item[$J$] ジョブの集合
  \item[$J_j \in J$] ジョブ$j$
  \item[$O_m$] オペレーション$m$
  \item[$ON_{j}$] ジョブ$j$に含まれるオペレーション数
  \item[$OAR_{jm} = \{r \in R\}$] ジョブ$j$のオペレーション$m$が利用可能なリソースの集合
  \item[$PT_{ijm}$] ジョブ$j$のオペレーション$m$がリソース$i$で実行される場合の処理時間
  \item[$OR_{jm} \in R$] ジョブ$j$のオペレーション$m$が割り当てられているリソース
\end{description}

今回議論する問題では、必ず$OAR_{jm}$は一意に決まっており、各ジョブの各オペレーションが実行されるリソースは事前に決まっているとする。

\cite{shuluo2020}では、下記の要素が状態から得られる特徴量として挙げられていた。

\begin{description}[style=multiline, leftmargin=10em]
  \item[$U_{ave}$] マシン$k$の稼働率を$U_k$としたときの、全マシンの$U_k$の平均値
  \item[$U_{std}$] 全マシンの$U_k$の標準偏差
  \item[$CRO_{ave}$] 全ジョブの操作終了数の平均値
  \item[$CRJ_{ave}$] 全ジョブの操作終了割合の平均値
  \item[$CRJ_{std}$] 全ジョブの操作終了割合の標準偏差
\end{description}

\noindent
残り2つの特徴量は、今回議論したい問題の場合、常に$0$となるので割愛する。
上記特徴量は、マシン・ジョブの区別がつかない特徴量なので、スケジューリング中の状態の変化に鈍感な箇所が存在する。上記特徴量に加えて以下の特徴量を追加する。

\begin{description}[style=multiline, leftmargin=10em]
  \item[$Type_{M_m}$] マシンのタイプ$m$
\end{description}

\begin{description}[style=multiline, leftmargin=15em]
  \item[$\bar{U_{k}} (Type_{M_k} = Type_{M_m})$] マシンタイプごとの$U_k$の平均値
  \item[$CPOC_j$] ジョブ$J_j$の終了した操作$PT_{ijm}$の操作時間の合計
  \item[$CPOR_j$] ジョブ$J_j$の未終了の操作$PT_{ijm}$の操作時間の合計
\end{description}

上記の特徴量の追加によって、$|R|+|J|\times2$個の特徴量が追加されることになる.

\begin{equation}
  f(x) = \left\{
    \begin{array}{ll}
    1 & (x \geq 0)\\
    0 & (x < 0)
    \end{array}
    \right.
\end{equation}

\begin{thebibliography}{99}
  \bibitem{chandrashekar2014} chandrashekar2014 \\
  \url{https://www.sciencedirect.com/science/article/abs/pii/S0045790613003066}

  \bibitem{shuluo2020} shuluo2020
\end{thebibliography}

\end{document}
