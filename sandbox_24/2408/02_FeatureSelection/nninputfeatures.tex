\documentclass{ltjsarticle}
\usepackage{enumitem}
\usepackage{url}

\begin{document}

\title{Neural Network Input Features}
\author{93nezyr}
\maketitle

\section{入力特長量生成・抽出}

ジョブショップスケジューリング問題(JSSP)における、特長量選択方法について議論する。今回のJSSPは、ガントチャートで表現されるものとする。ガントチャートの行方向をJSSPにおけるリソース(マシン)、列方向をスケジューリングの時刻とすると、JSSPの状態には以下の変数が含まれる。

\begin{description}[style=multiline, leftmargin=10em]
  \item[$t$] スケジューリング中の現在時刻
  \item[$R$] リソースの集合
  \item[$R_i \in R$] リソース$i$
  \item[$J$] ジョブの集合
  \item[$J_j \in J$] ジョブ$j$
  \item[$O_m$] オペレーション$m$
  \item[$ON_{j}$] ジョブ$j$に含まれるオペレーション数
  \item[$OAR_{jm} = \{r \in R\}$] ジョブ$j$のオペレーション$m$が利用可能なリソースの集合
  \item[$PT_{ijm}$] ジョブ$j$のオペレーション$m$がリソース$i$で実行される場合の処理時間
  \item[$OR_{jm} \in R$] ジョブ$j$のオペレーション$m$が割り当てられているリソース
\end{description}

今回議論する問題では、必ず$OAR_{jm}$は一意に決まっており、各ジョブの各オペレーションが実行されるリソースは事前に決まっているとする。

\begin{thebibliography}{99}
  \bibitem{chandrashekar2014} chandrashekar2014 \\
  \url{https://www.sciencedirect.com/science/article/abs/pii/S0045790613003066}

  \bibitem{citekey}
\end{thebibliography}

\end{document}
