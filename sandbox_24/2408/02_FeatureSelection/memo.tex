\documentclass{ltjsarticle}
\begin{document}

% おいおい文章中の文字をローマン体にするのは\textrm{\LaTeX}の基本だろう?
\title{わーいタイトルだっ\textrm{\LaTeX}}
\author{俺氏}
\maketitle
\section{見出しっ}
\subsection{小見出し!}
\paragraph{段落だー}
% ここも文章中だからtextrmでいいのでは?
% これはLua$\mathrm{\\LaTeX}$ でやってます!端で改行されるし、うぇええええええええええええええええええええええええええい\\
これはLua\textrm{\LaTeX} でやってます!端で改行されるし、うぇええええええええええええええええええええええええええい\\
強制改行もできます。aa
\end{document}
